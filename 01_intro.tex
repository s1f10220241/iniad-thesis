\chapter{はじめに}
本研究の目的は,象の卵を発見して,象の卵生を証明することである。進化論的には,象は卵を産
む方が自然である。世界の動物園や,アフリカ,インドで空と陸の両面から多角的に探索を行う。
象の卵を発見した場合は,その形状の測定,材質の解析,工学的応用の可能性の検討を行う。

湯川による研究~\cite{yukawa1950quantum}では...

\section{なぜ象は卵を産むはずか}
今まで,哺乳類である象は卵を産まないとされてきた。しかし,哺乳類の定義は乳を与える動物
のことであり,必ずしも胎盤を持ち母親の体内で成長させる動物であるとは限らない。たとえばカ
モノハシは卵を産むし,カンガルーは体外の袋の中で新生児を育てる。哺乳類の動物が胎生か卵胎
生か卵生かは,進化上の分類よりもむしろ,生活の環境によって決まる。象のように大きく強い動
物の場合,重たい象の胎児を運ぶよりは,卵を産んでその重さから解放される方が楽である。また
卵が大きく硬い殻でできていれば,他の動物に取られたり食べられたりする恐れもない。さらに食
物を求めて象の群れが移動するときも,長い鼻で丸い卵を転がして行った方が,胎児を持ち運ぶよ
りエネルギー効率が高い。(恐竜も卵を産んだが,長い鼻を持たず,車輪を考案するだけの脳を
持たなかったため,巣を作った)こうした点から,象は卵を産む方が進化論的に自然である。
