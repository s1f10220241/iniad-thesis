\documentclass[bachelor]{INIAD}%卒論用 
\addtolength{\footskip}{8mm}
\bibliographystyle{jplain} 
%\usepackage[dviout]{graphicx}
\usepackage[dvipdfmx]{graphicx}
\usepackage{bm}
\usepackage{amsmath}

%\usepackage{geometry}
%\geometry{left=30mm,right=30mm,top=35mm,bottom=30mm}

%\documentclass[oneside]{suribt}% 本文が * ページ以下のときに (掲示に注意)
\title{象の卵生についての研究}
%\titlewidth{}% タイトル幅 (指定するときは単位つきで)
\author{東洋 太郎}
\eauthor{Taro Toyo}% Copyright 表示で使われる
\studentid{1F99999999}
\supervisor{赤羽台 花子}% 1つの引数をとる (役職まで含めて書く)
%\supervisor{指導教員名 役職 \and 指導教員名 役職}% 複数教員の場合,\and でつなげる
\handin{2021}{1}% 提出月. 2 つ (年, 月) 引数をとる
%\keywords{キーワード1, キーワード2} % 概要の下に表示される
\renewcommand{\baselinestretch}{1.25}
\setcounter{tocdepth}{2}

\begin{document}
\mojiparline{40}
\maketitle%%%%%%%%%%%%%%%%%%% タイトル %%%%

\frontmatter% ここから前文

%\etitle{Title in English}

%\begin{eabstract}%%%%%%%%%%%%% 概要 %%%%%%%%
% 300 words abstract in English should be written here. 
%\end{eabstract}

\begin{abstract}%%%%%%%%%%%%% 概要 %%%%%%%%
 ここには論文要旨を記述します。論文要旨の書き方については、指導教員の指導を受けること!
\end{abstract}

%%%%%%%%%%%%% 目次 %%%%%%%%
{\makeatletter
\let\ps@jpl@in\ps@empty
\makeatother
\pagestyle{empty}
\tableofcontents
\clearpage}

\mainmatter% ここから本文 %%% 本文 %%%%%%%%

\include{01_intro.tex}     % はじめに
\include{02_related.tex}   % 関連研究
\include{03_proposed.tex}  % 提案手法

% 以降、実装や評価、結論などの章を適切に配置してください

\backmatter% ここから後付
\chapter{まとめ}

\section{本講義で学んだこと}

本講義では分散・並列のコンピューティングやKubernetesに基づくクラウド・オーケストレーションを学んだ。
分散・並列のコンピューティングではどのようにして効率よく計算などの実行結果を反映させるかを学んだ。
クラウドコンピューティングではDjangoで作られたアプリケーションをDockerイメージとしてパッケージ化したものの安定したシステムの実現を目指し、Kubernetesに基づくコンテナ・オーケストレーションを学んだ。

\section{本講義の感想}

感想としては、特に自分の中でクラウドコンピューティングの内容が自分の成長につながったと感じている。
元々コンピュータ・ソフトウェア演習を履修していたのでどうにもスライドの手順通りに画面が進まなく、自分で考えて作業する時間が多かったからだ。

\section{卒業研究に向けての抱負}



     % 第4章まとめ
\include{80_ack}           % 謝辞

\bibliography{thesis.bib}  % 参考文献

\appendix% ここから付録 %%%%% 付録 %%%%%%%
\include{90_appendix}      % 付録

\end{document}
